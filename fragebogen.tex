\documentclass[a4paper,11pt,BCOR10mm,egregdoesnotlikesansseriftitles,oneside,headsepline]{scrartcl}

\input{setup.tex}

\ihead{27. Februar 2024}
\ohead{\textbf{L1a} (LA-2023a)}
\chead{}
\cfoot{}

\begin{document}
\hyphenation{be-rück-sichtig-en}

\renewcommand{\QO}{$\ocircle$}% Use circles now instead of boxes.

\begin{center}
\vspace*{0.1cm}
\textbf{\huge Informatik -- Fragebogen zum Semesterstart}
\end{center}\vskip1em

\noindent Willkommen zum Fach Informatik!\hspace{0.1cm}  Bitte nehmen Sie sich 5--10 Min. Zeit, um diesen Fragebogen auszufüllen.
\vspace{0.2cm}

%\section*{About you}

\Qitem{ \Qq{\textbf{Vorname, Name} {\small (gemäss KFR-System)}}: \Qlinefill }

\Qitem{ \Qq{\textbf{Pronomen}} \hskip0.4cm \QO{} sie/ihr \hskip0.5cm \QO{} er/ihm \hskip0.5cm \QO{} \Qline{3cm} }

\Qitem{ \Qq{\textbf{\large Rufname} {\small (wie Sie von Lehrpersonen genannt werden möchten)}}: \Qlinefill }

\section*{Informatik-Kenntnisse}

%\Qitem{ \Qq{Listen Sie mindestens drei (3) Themen auf, die Ihrer Meinung nach zum
%Informatik-Unterricht gehören. Gibt es welche, die nicht auf der Benutzung eines Computers beruhen? \Qlines{2} }}

%\Qitem{ \Qq{Haben Sie schon mal etwas über Informatik gelernt? Falls ja, geben Sie
%bitte an, welche Kenntnisse Sie schon besitzten, und erläutern Sie wenn möglich, wie Sie diese
%Erfahrung gesammelt haben.
%\Qlines{4} }}
\Qitem{ \Qq{Geben Sie bitte an, welche Informatik-Kenntnisse Sie schon
besitzten.
\Qlines{8} }}

\section*{Erwartungen}
\minisec{Geben Sie an, wie stark Sie mit den folgenden Aussagen einverstanden sind}
\vskip.5em

\Qitem{ \Qq{Informatik interessiert mich} \Qtab{9cm}{überhaupt nicht \Qrating{5} sehr}}

\Qitem{ \Qq{Ich werde Informatik-Kenntnisse benötigen im Leben} \Qtab{9cm}{überhaupt nicht \Qrating{5} sehr}}

%\Qitem{ \Qq{Ich erwarte, gut in diesem Fach abzuschneiden} \Qtab{9cm}{überhaupt nicht \Qrating{5} sehr}}

\section*{Anmerkungen}

\Qitem{ \Qq{Falls Sie irgendwelche Anmerkungen haben, etwas, was die Lehrperson wissen oder beim Unterricht berücksichtigen sollte, können Sie dies hier erläutern:} \Qlines{4} }


\end{document}
